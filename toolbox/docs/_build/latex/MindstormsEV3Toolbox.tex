% Generated by Sphinx.
\def\sphinxdocclass{report}
\newif\ifsphinxKeepOldNames \sphinxKeepOldNamestrue
\documentclass[letterpaper,10pt,english]{sphinxmanual}
\usepackage{iftex}

\ifPDFTeX
  \usepackage[utf8]{inputenc}
\fi
\ifdefined\DeclareUnicodeCharacter
  \DeclareUnicodeCharacter{00A0}{\nobreakspace}
\fi
\usepackage{cmap}
\usepackage[T1]{fontenc}
\usepackage{amsmath,amssymb,amstext}
\usepackage{babel}
\usepackage{times}
\usepackage[Bjarne]{fncychap}
\usepackage{longtable}
\usepackage{sphinx}
\usepackage{multirow}
\usepackage{eqparbox}


\addto\captionsenglish{\renewcommand{\figurename}{Fig.\@ }}
\addto\captionsenglish{\renewcommand{\tablename}{Table }}
\SetupFloatingEnvironment{literal-block}{name=Listing }

\addto\extrasenglish{\def\pageautorefname{page}}

\setcounter{tocdepth}{3}


\title{Mindstorms EV3 Toolbox Documentation}
\date{Dec 08, 2016}
\release{v0.3-rc.8}
\author{Tim Stadtmann}
\newcommand{\sphinxlogo}{}
\renewcommand{\releasename}{Release}
\makeindex

\makeatletter
\def\PYG@reset{\let\PYG@it=\relax \let\PYG@bf=\relax%
    \let\PYG@ul=\relax \let\PYG@tc=\relax%
    \let\PYG@bc=\relax \let\PYG@ff=\relax}
\def\PYG@tok#1{\csname PYG@tok@#1\endcsname}
\def\PYG@toks#1+{\ifx\relax#1\empty\else%
    \PYG@tok{#1}\expandafter\PYG@toks\fi}
\def\PYG@do#1{\PYG@bc{\PYG@tc{\PYG@ul{%
    \PYG@it{\PYG@bf{\PYG@ff{#1}}}}}}}
\def\PYG#1#2{\PYG@reset\PYG@toks#1+\relax+\PYG@do{#2}}

\expandafter\def\csname PYG@tok@gd\endcsname{\def\PYG@tc##1{\textcolor[rgb]{0.63,0.00,0.00}{##1}}}
\expandafter\def\csname PYG@tok@gu\endcsname{\let\PYG@bf=\textbf\def\PYG@tc##1{\textcolor[rgb]{0.50,0.00,0.50}{##1}}}
\expandafter\def\csname PYG@tok@gt\endcsname{\def\PYG@tc##1{\textcolor[rgb]{0.00,0.27,0.87}{##1}}}
\expandafter\def\csname PYG@tok@gs\endcsname{\let\PYG@bf=\textbf}
\expandafter\def\csname PYG@tok@gr\endcsname{\def\PYG@tc##1{\textcolor[rgb]{1.00,0.00,0.00}{##1}}}
\expandafter\def\csname PYG@tok@cm\endcsname{\let\PYG@it=\textit\def\PYG@tc##1{\textcolor[rgb]{0.25,0.50,0.56}{##1}}}
\expandafter\def\csname PYG@tok@vg\endcsname{\def\PYG@tc##1{\textcolor[rgb]{0.73,0.38,0.84}{##1}}}
\expandafter\def\csname PYG@tok@vi\endcsname{\def\PYG@tc##1{\textcolor[rgb]{0.73,0.38,0.84}{##1}}}
\expandafter\def\csname PYG@tok@mh\endcsname{\def\PYG@tc##1{\textcolor[rgb]{0.13,0.50,0.31}{##1}}}
\expandafter\def\csname PYG@tok@cs\endcsname{\def\PYG@tc##1{\textcolor[rgb]{0.25,0.50,0.56}{##1}}\def\PYG@bc##1{\setlength{\fboxsep}{0pt}\colorbox[rgb]{1.00,0.94,0.94}{\strut ##1}}}
\expandafter\def\csname PYG@tok@ge\endcsname{\let\PYG@it=\textit}
\expandafter\def\csname PYG@tok@vc\endcsname{\def\PYG@tc##1{\textcolor[rgb]{0.73,0.38,0.84}{##1}}}
\expandafter\def\csname PYG@tok@il\endcsname{\def\PYG@tc##1{\textcolor[rgb]{0.13,0.50,0.31}{##1}}}
\expandafter\def\csname PYG@tok@go\endcsname{\def\PYG@tc##1{\textcolor[rgb]{0.20,0.20,0.20}{##1}}}
\expandafter\def\csname PYG@tok@cp\endcsname{\def\PYG@tc##1{\textcolor[rgb]{0.00,0.44,0.13}{##1}}}
\expandafter\def\csname PYG@tok@gi\endcsname{\def\PYG@tc##1{\textcolor[rgb]{0.00,0.63,0.00}{##1}}}
\expandafter\def\csname PYG@tok@gh\endcsname{\let\PYG@bf=\textbf\def\PYG@tc##1{\textcolor[rgb]{0.00,0.00,0.50}{##1}}}
\expandafter\def\csname PYG@tok@ni\endcsname{\let\PYG@bf=\textbf\def\PYG@tc##1{\textcolor[rgb]{0.84,0.33,0.22}{##1}}}
\expandafter\def\csname PYG@tok@nl\endcsname{\let\PYG@bf=\textbf\def\PYG@tc##1{\textcolor[rgb]{0.00,0.13,0.44}{##1}}}
\expandafter\def\csname PYG@tok@nn\endcsname{\let\PYG@bf=\textbf\def\PYG@tc##1{\textcolor[rgb]{0.05,0.52,0.71}{##1}}}
\expandafter\def\csname PYG@tok@no\endcsname{\def\PYG@tc##1{\textcolor[rgb]{0.38,0.68,0.84}{##1}}}
\expandafter\def\csname PYG@tok@na\endcsname{\def\PYG@tc##1{\textcolor[rgb]{0.25,0.44,0.63}{##1}}}
\expandafter\def\csname PYG@tok@nb\endcsname{\def\PYG@tc##1{\textcolor[rgb]{0.00,0.44,0.13}{##1}}}
\expandafter\def\csname PYG@tok@nc\endcsname{\let\PYG@bf=\textbf\def\PYG@tc##1{\textcolor[rgb]{0.05,0.52,0.71}{##1}}}
\expandafter\def\csname PYG@tok@nd\endcsname{\let\PYG@bf=\textbf\def\PYG@tc##1{\textcolor[rgb]{0.33,0.33,0.33}{##1}}}
\expandafter\def\csname PYG@tok@ne\endcsname{\def\PYG@tc##1{\textcolor[rgb]{0.00,0.44,0.13}{##1}}}
\expandafter\def\csname PYG@tok@nf\endcsname{\def\PYG@tc##1{\textcolor[rgb]{0.02,0.16,0.49}{##1}}}
\expandafter\def\csname PYG@tok@si\endcsname{\let\PYG@it=\textit\def\PYG@tc##1{\textcolor[rgb]{0.44,0.63,0.82}{##1}}}
\expandafter\def\csname PYG@tok@s2\endcsname{\def\PYG@tc##1{\textcolor[rgb]{0.25,0.44,0.63}{##1}}}
\expandafter\def\csname PYG@tok@nt\endcsname{\let\PYG@bf=\textbf\def\PYG@tc##1{\textcolor[rgb]{0.02,0.16,0.45}{##1}}}
\expandafter\def\csname PYG@tok@nv\endcsname{\def\PYG@tc##1{\textcolor[rgb]{0.73,0.38,0.84}{##1}}}
\expandafter\def\csname PYG@tok@s1\endcsname{\def\PYG@tc##1{\textcolor[rgb]{0.25,0.44,0.63}{##1}}}
\expandafter\def\csname PYG@tok@ch\endcsname{\let\PYG@it=\textit\def\PYG@tc##1{\textcolor[rgb]{0.25,0.50,0.56}{##1}}}
\expandafter\def\csname PYG@tok@m\endcsname{\def\PYG@tc##1{\textcolor[rgb]{0.13,0.50,0.31}{##1}}}
\expandafter\def\csname PYG@tok@gp\endcsname{\let\PYG@bf=\textbf\def\PYG@tc##1{\textcolor[rgb]{0.78,0.36,0.04}{##1}}}
\expandafter\def\csname PYG@tok@sh\endcsname{\def\PYG@tc##1{\textcolor[rgb]{0.25,0.44,0.63}{##1}}}
\expandafter\def\csname PYG@tok@ow\endcsname{\let\PYG@bf=\textbf\def\PYG@tc##1{\textcolor[rgb]{0.00,0.44,0.13}{##1}}}
\expandafter\def\csname PYG@tok@sx\endcsname{\def\PYG@tc##1{\textcolor[rgb]{0.78,0.36,0.04}{##1}}}
\expandafter\def\csname PYG@tok@bp\endcsname{\def\PYG@tc##1{\textcolor[rgb]{0.00,0.44,0.13}{##1}}}
\expandafter\def\csname PYG@tok@c1\endcsname{\let\PYG@it=\textit\def\PYG@tc##1{\textcolor[rgb]{0.25,0.50,0.56}{##1}}}
\expandafter\def\csname PYG@tok@o\endcsname{\def\PYG@tc##1{\textcolor[rgb]{0.40,0.40,0.40}{##1}}}
\expandafter\def\csname PYG@tok@kc\endcsname{\let\PYG@bf=\textbf\def\PYG@tc##1{\textcolor[rgb]{0.00,0.44,0.13}{##1}}}
\expandafter\def\csname PYG@tok@c\endcsname{\let\PYG@it=\textit\def\PYG@tc##1{\textcolor[rgb]{0.25,0.50,0.56}{##1}}}
\expandafter\def\csname PYG@tok@mf\endcsname{\def\PYG@tc##1{\textcolor[rgb]{0.13,0.50,0.31}{##1}}}
\expandafter\def\csname PYG@tok@err\endcsname{\def\PYG@bc##1{\setlength{\fboxsep}{0pt}\fcolorbox[rgb]{1.00,0.00,0.00}{1,1,1}{\strut ##1}}}
\expandafter\def\csname PYG@tok@mb\endcsname{\def\PYG@tc##1{\textcolor[rgb]{0.13,0.50,0.31}{##1}}}
\expandafter\def\csname PYG@tok@ss\endcsname{\def\PYG@tc##1{\textcolor[rgb]{0.32,0.47,0.09}{##1}}}
\expandafter\def\csname PYG@tok@sr\endcsname{\def\PYG@tc##1{\textcolor[rgb]{0.14,0.33,0.53}{##1}}}
\expandafter\def\csname PYG@tok@mo\endcsname{\def\PYG@tc##1{\textcolor[rgb]{0.13,0.50,0.31}{##1}}}
\expandafter\def\csname PYG@tok@kd\endcsname{\let\PYG@bf=\textbf\def\PYG@tc##1{\textcolor[rgb]{0.00,0.44,0.13}{##1}}}
\expandafter\def\csname PYG@tok@mi\endcsname{\def\PYG@tc##1{\textcolor[rgb]{0.13,0.50,0.31}{##1}}}
\expandafter\def\csname PYG@tok@kn\endcsname{\let\PYG@bf=\textbf\def\PYG@tc##1{\textcolor[rgb]{0.00,0.44,0.13}{##1}}}
\expandafter\def\csname PYG@tok@cpf\endcsname{\let\PYG@it=\textit\def\PYG@tc##1{\textcolor[rgb]{0.25,0.50,0.56}{##1}}}
\expandafter\def\csname PYG@tok@kr\endcsname{\let\PYG@bf=\textbf\def\PYG@tc##1{\textcolor[rgb]{0.00,0.44,0.13}{##1}}}
\expandafter\def\csname PYG@tok@s\endcsname{\def\PYG@tc##1{\textcolor[rgb]{0.25,0.44,0.63}{##1}}}
\expandafter\def\csname PYG@tok@kp\endcsname{\def\PYG@tc##1{\textcolor[rgb]{0.00,0.44,0.13}{##1}}}
\expandafter\def\csname PYG@tok@w\endcsname{\def\PYG@tc##1{\textcolor[rgb]{0.73,0.73,0.73}{##1}}}
\expandafter\def\csname PYG@tok@kt\endcsname{\def\PYG@tc##1{\textcolor[rgb]{0.56,0.13,0.00}{##1}}}
\expandafter\def\csname PYG@tok@sc\endcsname{\def\PYG@tc##1{\textcolor[rgb]{0.25,0.44,0.63}{##1}}}
\expandafter\def\csname PYG@tok@sb\endcsname{\def\PYG@tc##1{\textcolor[rgb]{0.25,0.44,0.63}{##1}}}
\expandafter\def\csname PYG@tok@k\endcsname{\let\PYG@bf=\textbf\def\PYG@tc##1{\textcolor[rgb]{0.00,0.44,0.13}{##1}}}
\expandafter\def\csname PYG@tok@se\endcsname{\let\PYG@bf=\textbf\def\PYG@tc##1{\textcolor[rgb]{0.25,0.44,0.63}{##1}}}
\expandafter\def\csname PYG@tok@sd\endcsname{\let\PYG@it=\textit\def\PYG@tc##1{\textcolor[rgb]{0.25,0.44,0.63}{##1}}}

\def\PYGZbs{\char`\\}
\def\PYGZus{\char`\_}
\def\PYGZob{\char`\{}
\def\PYGZcb{\char`\}}
\def\PYGZca{\char`\^}
\def\PYGZam{\char`\&}
\def\PYGZlt{\char`\<}
\def\PYGZgt{\char`\>}
\def\PYGZsh{\char`\#}
\def\PYGZpc{\char`\%}
\def\PYGZdl{\char`\$}
\def\PYGZhy{\char`\-}
\def\PYGZsq{\char`\'}
\def\PYGZdq{\char`\"}
\def\PYGZti{\char`\~}
% for compatibility with earlier versions
\def\PYGZat{@}
\def\PYGZlb{[}
\def\PYGZrb{]}
\makeatother

\renewcommand\PYGZsq{\textquotesingle}

\begin{document}

\maketitle
\tableofcontents
\phantomsection\label{index::doc}


Contents:
\phantomsection\label{source:module-source}\index{source (module)}

\chapter{EV3}
\label{source:ev3}\label{source::doc}\label{source:toolbox-for-controlling-lego-mindstorms-ev3-with-matlab}\index{EV3 (class in source)}

\begin{fulllineitems}
\phantomsection\label{source:source.EV3}\pysiglinewithargsret{\sphinxstrong{class }\sphinxcode{source.}\sphinxbfcode{EV3}}{\emph{varargin}}{}
High-level class to work with physical bricks.

This is the `central' class (from user's view) when working with this toolbox. It
delivers a convenient interface for creating a connection to the brick and sending
commands to it. An EV3-object creates 4 Motor- and 4 Sensor-objects, one for each port.
\paragraph{Notes}
\begin{itemize}
\item {} 
Creating multiple EV3 objects and connecting them to different physical bricks has not
been thoroughly tested yet, but seems to work on a first glance.

\end{itemize}
\index{motorA (source.EV3 attribute)}

\begin{fulllineitems}
\phantomsection\label{source:source.EV3.motorA}\pysigline{\sphinxbfcode{motorA}}
\emph{Motor} -- Motor-object interfacing port A

\end{fulllineitems}

\index{motorB (source.EV3 attribute)}

\begin{fulllineitems}
\phantomsection\label{source:source.EV3.motorB}\pysigline{\sphinxbfcode{motorB}}
\emph{Motor} -- Motor-object interfacing port B

\end{fulllineitems}

\index{motorC (source.EV3 attribute)}

\begin{fulllineitems}
\phantomsection\label{source:source.EV3.motorC}\pysigline{\sphinxbfcode{motorC}}
\emph{Motor} -- Motor-object interfacing port C

\end{fulllineitems}

\index{motorD (source.EV3 attribute)}

\begin{fulllineitems}
\phantomsection\label{source:source.EV3.motorD}\pysigline{\sphinxbfcode{motorD}}
\emph{Motor} -- Motor-object interfacing port D

\end{fulllineitems}

\index{sensor1 (source.EV3 attribute)}

\begin{fulllineitems}
\phantomsection\label{source:source.EV3.sensor1}\pysigline{\sphinxbfcode{sensor1}}
\emph{Sensor} -- Motor-object interfacing port 1

\end{fulllineitems}

\index{sensor2 (source.EV3 attribute)}

\begin{fulllineitems}
\phantomsection\label{source:source.EV3.sensor2}\pysigline{\sphinxbfcode{sensor2}}
\emph{Sensor} -- Motor-object interfacing port 2

\end{fulllineitems}

\index{sensor3 (source.EV3 attribute)}

\begin{fulllineitems}
\phantomsection\label{source:source.EV3.sensor3}\pysigline{\sphinxbfcode{sensor3}}
\emph{Sensor} -- Motor-object interfacing port 3

\end{fulllineitems}

\index{sensor4 (source.EV3 attribute)}

\begin{fulllineitems}
\phantomsection\label{source:source.EV3.sensor4}\pysigline{\sphinxbfcode{sensor4}}
\emph{Sensor} -- Motor-object interfacing port 4

\end{fulllineitems}

\index{debug (source.EV3 attribute)}

\begin{fulllineitems}
\phantomsection\label{source:source.EV3.debug}\pysigline{\sphinxbfcode{debug}}
\emph{numeric in \{0,1,2\}} -- Debug mode. \emph{{[}WRITABLE{]}}
\begin{itemize}
\item {} 
0: Debug turned off

\item {} 
1: Debug turned on for EV3-object -\textgreater{} enables feedback in the console about what firmware-commands have been called when using a method

\item {} 
2: Low-level-Debug turned on -\textgreater{} each packet sent and received is printed to the console

\end{itemize}

\end{fulllineitems}

\index{batteryMode (source.EV3 attribute)}

\begin{fulllineitems}
\phantomsection\label{source:source.EV3.batteryMode}\pysigline{\sphinxbfcode{batteryMode}}
\emph{string in \{`Percentage', `Voltage'\}} -- Mode for reading battery charge.
\emph{{[}WRITABLE{]}}

\end{fulllineitems}

\index{batteryValue (source.EV3 attribute)}

\begin{fulllineitems}
\phantomsection\label{source:source.EV3.batteryValue}\pysigline{\sphinxbfcode{batteryValue}}
\emph{numeric} -- Current battery charge. Depending on batteryMode, the reading
is either in percentage or voltage. \emph{{[}READ-ONLY{]}}

\end{fulllineitems}

\index{isConnected (source.EV3 attribute)}

\begin{fulllineitems}
\phantomsection\label{source:source.EV3.isConnected}\pysigline{\sphinxbfcode{isConnected}}
\emph{bool} -- True if virtual brick-object is connected to physical one. \emph{{[}READ-ONLY{]}}

\end{fulllineitems}

\paragraph{Examples}

b = EV3(); 
b.connect(`usb'); 
ma = b.motorA; 
ma.setProperties(`power', 50, `limitValue', 720); 
ma.start(); 
\% fun 
b.sensor1.value 
b.waitFor(); 
b.beep(); 
delete b; 
\index{beep() (source.EV3 method)}

\begin{fulllineitems}
\phantomsection\label{source:source.EV3.beep}\pysiglinewithargsret{\sphinxbfcode{beep}}{\emph{ev3}}{}
Plays a `beep'-tone on brick.
\paragraph{Notes}
\begin{itemize}
\item {} 
This equals playTone(10, 1000, 100) (Wraps the same opCode in comm-layer)

\end{itemize}
\paragraph{Example}

b = EV3(); 
b.connect(`bt', `serPort', `/dev/rfcomm0'); 
b.beep(); 

\end{fulllineitems}

\index{connect() (source.EV3 method)}

\begin{fulllineitems}
\phantomsection\label{source:source.EV3.connect}\pysiglinewithargsret{\sphinxbfcode{connect}}{\emph{ev3}, \emph{varargin}}{}
Connects EV3-object and its Motors and Sensors to physical brick.
\begin{quote}\begin{description}
\item[{Parameters}] \leavevmode\begin{itemize}
\item {} 
\textbf{\texttt{connectionType}} (\emph{\texttt{string in \{'bt', 'usb'\}}}) -- Connection type

\item {} 
\textbf{\texttt{serPort}} (\emph{\texttt{string in \{'/dev/rfcomm1', '/dev/rfcomm2', ...\}}}) -- Path to serial port
(if `bt')

\item {} 
\textbf{\texttt{beep}} (\emph{\texttt{bool}}) -- If true, EV3 beeps if connection has been established

\end{itemize}

\end{description}\end{quote}
\paragraph{Examples}

\% Setup bluetooth connection via com-port 0 
b = EV3(); 
b.connect(`bt', `serPort', `/dev/rfcomm0'); 
\% Setup usb connection, beep when connection has been established
b = EV3(); 
b.connect(`usb', `beep', `on', ); 

Check connection

\end{fulllineitems}

\index{disconnect() (source.EV3 method)}

\begin{fulllineitems}
\phantomsection\label{source:source.EV3.disconnect}\pysiglinewithargsret{\sphinxbfcode{disconnect}}{\emph{ev3}}{}
Disconnects EV3-object and its Motors and Sensors from physical brick.
\paragraph{Notes}
\begin{itemize}
\item {} 
Gets called automatically when EV3-object is destroyed.

\end{itemize}
\paragraph{Example}

b = EV3();
b.connect(`bt', `serPort', `/dev/rfcomm0');
\% do stuff
b.disconnect();

Reset motors and sensors before disconnecting

\end{fulllineitems}

\index{playTone() (source.EV3 method)}

\begin{fulllineitems}
\phantomsection\label{source:source.EV3.playTone}\pysiglinewithargsret{\sphinxbfcode{playTone}}{\emph{ev3}, \emph{volume}, \emph{frequency}, \emph{duration}}{}
Plays tone on brick.
\begin{quote}\begin{description}
\item[{Parameters}] \leavevmode\begin{itemize}
\item {} 
\textbf{\texttt{volume}} (\emph{\texttt{numeric in {[}0, 100{]}}}) -- in percent

\item {} 
\textbf{\texttt{frequency}} (\emph{\texttt{numeric in {[}250, 10000{]}}}) -- in Hertz

\item {} 
\textbf{\texttt{duration}} (\emph{\texttt{numeric \textgreater{}0}}) -- in milliseconds

\end{itemize}

\end{description}\end{quote}
\paragraph{Example}

b = EV3(); 
b.connect(`bt', `serPort', `/dev/rfcomm0'); 
b.playTone(50, 5000, 1000);  \% Plays tone with 50\% volume and 5000Hz for 1
second. 

\end{fulllineitems}

\index{setProperties() (source.EV3 method)}

\begin{fulllineitems}
\phantomsection\label{source:source.EV3.setProperties}\pysiglinewithargsret{\sphinxbfcode{setProperties}}{\emph{ev3}, \emph{varargin}}{}
Set multiple EV3 properties at once using MATLAB's inputParser.
\begin{quote}\begin{description}
\item[{Parameters}] \leavevmode\begin{itemize}
\item {} 
\textbf{\texttt{debug}} (\emph{\texttt{numeric in \{0,1,2\}}}) -- see EV3.debug \emph{{[}OPTIONAL{]}}

\item {} 
\textbf{\texttt{batteryMode}} (\emph{\texttt{string in \{'Voltage'/'Percentage'\}}}) -- see EV3.batteryMode \emph{{[}OPTIONAL{]}}

\end{itemize}

\end{description}\end{quote}
\paragraph{Example}

b = EV3(); 
b.connect(`bt', `serPort', `/dev/rfcomm0'); 
b.setProperties(`debug', `on', `batteryMode', `Voltage'); 
\% Instead of: b.debug = `on'; b.batteryMode = `Voltage'; 

See also EV3.DEBUG, EV3.BATTERYMODE

\end{fulllineitems}

\index{stopAllMotors() (source.EV3 method)}

\begin{fulllineitems}
\phantomsection\label{source:source.EV3.stopAllMotors}\pysiglinewithargsret{\sphinxbfcode{stopAllMotors}}{\emph{ev3}}{}
Sends a stop-command to all motor-ports

\end{fulllineitems}

\index{stopTone() (source.EV3 method)}

\begin{fulllineitems}
\phantomsection\label{source:source.EV3.stopTone}\pysiglinewithargsret{\sphinxbfcode{stopTone}}{\emph{ev3}}{}
Stops tone currently played
\paragraph{Example}

b = EV3(); 
b.connect(`bt', `serPort', `/dev/rfcomm0'); 
b.playTone(10,100,100000000);  \% Accidentally given wrong tone duration :) 
b.stopTone();  \% Stops tone immediately. 

\end{fulllineitems}

\index{tonePlayed() (source.EV3 method)}

\begin{fulllineitems}
\phantomsection\label{source:source.EV3.tonePlayed}\pysiglinewithargsret{\sphinxbfcode{tonePlayed}}{\emph{ev3}}{}
Tests if tone is currently played.
\begin{quote}\begin{description}
\item[{Returns}] \leavevmode
\textbf{status} -- True if a tone is being played

\item[{Return type}] \leavevmode
bool

\end{description}\end{quote}
\begin{description}
\item[{Example}] \leavevmode
b = EV3(); 
b.connect(`bt', `serPort', `/dev/rfcomm0'); 
b.playTone(10, 100, 1000); 
pause(0.5); 
b.tonePlayed() -\textgreater{} Outputs 1 to console. 

\end{description}

\end{fulllineitems}


\end{fulllineitems}



\chapter{Motor}
\label{source:motor}\index{Motor (class in source)}

\begin{fulllineitems}
\phantomsection\label{source:source.Motor}\pysiglinewithargsret{\sphinxstrong{class }\sphinxcode{source.}\sphinxbfcode{Motor}}{\emph{varargin}}{}
High-level class to work with motors.

This class is supposed to ease the use of the brick's motors. It is possible to set all
kinds of parameters, request the current status of the motor ports and of course send
commands to the brick to be executed on the respective port.
\paragraph{Notes}
\begin{itemize}
\item {} 
You don't need to create instances of this class. The EV3-class automatically creates
instances for each motor port, and you can work with them via the EV3-object.

\item {} 
The Motor-class represents motor ports, not individual motors!

\end{itemize}
\index{power (source.Motor attribute)}

\begin{fulllineitems}
\phantomsection\label{source:source.Motor.power}\pysigline{\sphinxbfcode{power}}
\emph{numeric in {[}-100, 100{]}} -- Power level of motor in percent. \emph{{[}WRITABLE{]}}

\end{fulllineitems}

\index{speedRegulation (source.Motor attribute)}

\begin{fulllineitems}
\phantomsection\label{source:source.Motor.speedRegulation}\pysigline{\sphinxbfcode{speedRegulation}}
\emph{bool} -- Speed regulation turned on or off. When turned on, motor will
try to `hold' its speed at given power level, whatever the load. In this mode, the
highest possible speed depends on the load and mostly goes up to around 70-80 (at
this point, the Brick internally input 100\% power). When turned off, motor will
constantly input the same power into the motor. The resulting speed will be
somewhat lower, depending on the load. \emph{{[}WRITABLE{]}}

\end{fulllineitems}

\index{smoothStart (source.Motor attribute)}

\begin{fulllineitems}
\phantomsection\label{source:source.Motor.smoothStart}\pysigline{\sphinxbfcode{smoothStart}}
\emph{numeric s. t. smoothStart+smoothStop \textless{} limitValue} -- Degrees/Time
indicating how far/long the motor should smoothly start. Depending on limitMode,
the input is interpreted either in degrees or milliseconds. The first
\{smoothStart\}-milliseconds/degrees of limitValue the motor will slowly accelerate
until reaching its defined speed. \emph{{[}WRITABLE{]}}

\end{fulllineitems}

\index{smoothStop (source.Motor attribute)}

\begin{fulllineitems}
\phantomsection\label{source:source.Motor.smoothStop}\pysigline{\sphinxbfcode{smoothStop}}
\emph{numeric s. t. smoothStart+smoothStop \textless{} limitValue} -- Degrees/Time
indicating how far/long the motor should smoothly stop. Depending on limitMode, the
input is interpreted either in degrees or milliseconds. The last
{[}smoothStop{]}-milliseconds/degrees of limitValue the motor will slowly slow down
until it has stopped. \emph{{[}WRITABLE{]}}

\end{fulllineitems}

\index{limitValue (source.Motor attribute)}

\begin{fulllineitems}
\phantomsection\label{source:source.Motor.limitValue}\pysigline{\sphinxbfcode{limitValue}}
\emph{numeric\textgreater{}=0} -- Degrees/Time indicating how far/long the motor should run.
Depending on limitMode, the input is interpreted either in degrees or
milliseconds. \emph{{[}WRITABLE{]}}

\end{fulllineitems}

\index{limitMode (source.Motor attribute)}

\begin{fulllineitems}
\phantomsection\label{source:source.Motor.limitMode}\pysigline{\sphinxbfcode{limitMode}}
\emph{`Tacho'\textbar{}'Time'} -- Mode for motor limit. \emph{{[}WRITABLE{]}}

\end{fulllineitems}

\index{brakeMode (source.Motor attribute)}

\begin{fulllineitems}
\phantomsection\label{source:source.Motor.brakeMode}\pysigline{\sphinxbfcode{brakeMode}}
\emph{`Brake'\textbar{}'Coast'} -- Mode for braking. If `Coast', the motor will (at
tacholimit, if \textasciitilde{}=0) coast to a stop. If `Brake', the motor will stop immediately
(at tacholimit, if \textasciitilde{}=0) and hold the brake. \emph{{[}WRITABLE{]}}

\end{fulllineitems}

\index{debug (source.Motor attribute)}

\begin{fulllineitems}
\phantomsection\label{source:source.Motor.debug}\pysigline{\sphinxbfcode{debug}}
\emph{bool} -- Debug turned on or off. In debug mode, everytime a command is passed to
the sublayer (`communication layer'), there is feedback in the console about what
command has been called. \emph{{[}WRITABLE{]}}

\end{fulllineitems}

\index{isRunning (source.Motor attribute)}

\begin{fulllineitems}
\phantomsection\label{source:source.Motor.isRunning}\pysigline{\sphinxbfcode{isRunning}}
\emph{bool} -- True if motor is running. \emph{{[}READ-ONLY{]}}

\end{fulllineitems}

\index{tachoCount (source.Motor attribute)}

\begin{fulllineitems}
\phantomsection\label{source:source.Motor.tachoCount}\pysigline{\sphinxbfcode{tachoCount}}
\emph{numeric} -- Current tacho count. \emph{{[}READ-ONLY{]}}

\end{fulllineitems}

\index{currentSpeed (source.Motor attribute)}

\begin{fulllineitems}
\phantomsection\label{source:source.Motor.currentSpeed}\pysigline{\sphinxbfcode{currentSpeed}}
\emph{numeric} -- Current speed of motor. If speedRegulation=on this should equal power,
otherwise it will probably be lower than that. \emph{{[}READ-ONLY{]}}

\end{fulllineitems}

\index{type (source.Motor attribute)}

\begin{fulllineitems}
\phantomsection\label{source:source.Motor.type}\pysigline{\sphinxbfcode{type}}
\emph{DeviceType} -- Type of connected device if any. \emph{{[}READ-ONLY{]}}

\end{fulllineitems}

\index{internalReset() (source.Motor method)}

\begin{fulllineitems}
\phantomsection\label{source:source.Motor.internalReset}\pysiglinewithargsret{\sphinxbfcode{internalReset}}{\emph{motor}}{}
Resets internal tacho count. Use this if motor behaves weird (i.e. not starting at all, or not correctly
running to limitValue)

The internal tacho count is used for positioning the motor. When the
motor is running with a tacho limit, internally it uses another counter than the
one read by tachoCount. This internal tacho count needs to be reset if you
physically change the motor's position or it coasted into a stop. If the motor's
brakemode is `Coast', this function is called automatically.
\paragraph{Notes}
\begin{itemize}
\item {} 
A better name would probably be resetPosition...

\end{itemize}

See also MOTOR.RESETTACHOCOUNT

\end{fulllineitems}

\index{resetTachoCount() (source.Motor method)}

\begin{fulllineitems}
\phantomsection\label{source:source.Motor.resetTachoCount}\pysiglinewithargsret{\sphinxbfcode{resetTachoCount}}{\emph{motor}}{}
Resets tachocount

\end{fulllineitems}

\index{setBrake() (source.Motor method)}

\begin{fulllineitems}
\phantomsection\label{source:source.Motor.setBrake}\pysiglinewithargsret{\sphinxbfcode{setBrake}}{\emph{motor}, \emph{brake}}{}
Apply or release brake of motor
\begin{quote}\begin{description}
\item[{Parameters}] \leavevmode
\textbf{\texttt{brake}} (\emph{\texttt{bool}}) -- If true, brake will be pulled

\end{description}\end{quote}

\end{fulllineitems}

\index{setProperties() (source.Motor method)}

\begin{fulllineitems}
\phantomsection\label{source:source.Motor.setProperties}\pysiglinewithargsret{\sphinxbfcode{setProperties}}{\emph{motor}, \emph{varargin}}{}
Sets multiple Motor properties at once using MATLAB's inputParser.
\begin{quote}\begin{description}
\item[{Parameters}] \leavevmode\begin{itemize}
\item {} 
\textbf{\texttt{debug}} (\emph{\texttt{bool}}) -- 

\item {} 
\textbf{\texttt{smoothStart}} (\emph{\texttt{numeric in {[}0, limitValue{]}}}) -- 

\item {} 
\textbf{\texttt{smoothStop}} (\emph{\texttt{numeric in {[}0, limitValue{]}}}) -- 

\item {} 
\textbf{\texttt{speedRegulation}} (\emph{\texttt{bool}}) -- 

\item {} 
\textbf{\texttt{brakeMode}} (\emph{\texttt{'Coast'\textbar{}'Brake'}}) -- 

\item {} 
\textbf{\texttt{limitMode}} (\emph{\texttt{'Time'\textbar{}'Tacho'}}) -- 

\item {} 
\textbf{\texttt{limitValue}} (\emph{\texttt{numeric \textgreater{} 0}}) -- 

\item {} 
\textbf{\texttt{power}} (\emph{\texttt{numeric in {[}-100,100{]}}}) -- 

\item {} 
\textbf{\texttt{batteryMode}} (\emph{\texttt{'Voltage'\textbar{}'Percentage'}}) -- 

\end{itemize}

\end{description}\end{quote}
\paragraph{Example}

b = EV3(); 
b.connect(`bt', `serPort', `/dev/rfcomm0'); 
b.motorA.setProperties(`debug', `on', `power', 50, `limitValue', 720, `speedRegulation', `on'); 
\% Instead of: b.motorA.debug = `on'; 
\%             b.motorA.power = 50; 
\%             b.motorA.limitValue = 720; 
\%             b.motorA.speedRegulation = `on'; 

\end{fulllineitems}

\index{start() (source.Motor method)}

\begin{fulllineitems}
\phantomsection\label{source:source.Motor.start}\pysiglinewithargsret{\sphinxbfcode{start}}{\emph{motor}}{}
Starts the motor
\paragraph{Notes}
\begin{itemize}
\item {} 
Right now, alternatingly calling this function with and without tacho limit
may lead to unexpected behaviour. For example, if you run the motor without
a tacholimit for some time using Coast, then stop using Coast, and then try
to run the with a tacholimit, it will stop sooner or later than expected,
or may not even start at all.

\item {} 
After calling one of the functions to control the motor with some kind of
limit (which is done if limit\textasciitilde{}=0), the physical brick's power/speed value for
starting without a limit (i.e. if limit==0) is reset to zero. So if you want
to control the motor without a limit after doing so with a limit, you would
have to set the power manually to the desired value again. (I don't really
know if this is deliberate or a bug, and at this point, I'm too afraid to ask.)
To avoid confusion, this is done automatically in this special case.
However, this does not even work all the time. If motor does not
start, call stop() and setPower() manually. :/

\end{itemize}

Check connection and if motor is already running

\end{fulllineitems}

\index{stop() (source.Motor method)}

\begin{fulllineitems}
\phantomsection\label{source:source.Motor.stop}\pysiglinewithargsret{\sphinxbfcode{stop}}{\emph{motor}}{}
Stops the motor

\end{fulllineitems}

\index{syncedStart() (source.Motor method)}

\begin{fulllineitems}
\phantomsection\label{source:source.Motor.syncedStart}\pysiglinewithargsret{\sphinxbfcode{syncedStart}}{\emph{motor}, \emph{syncMotor}, \emph{varargin}}{}
Starts this motor synchronized with another

This motor acts as a `master', meaning that the synchronized control is done via
this one. When syncedStart is called, the master sets some of the slave's
(syncMotor) properties to keep it consistent with the physical brick. So, for
example, changing the power on the master motor will take effect
on the slave as soon as this method is called.
The following parameters will be affected on the slave: power, brakeMode,
limitValue, speedRegulation
\begin{quote}\begin{description}
\item[{Parameters}] \leavevmode\begin{itemize}
\item {} 
\textbf{\texttt{syncMotor}} (\emph{\texttt{Motor}}) -- the motor-object to sync with

\item {} 
\textbf{\texttt{turnRatio}} (\emph{\texttt{numeric in {[}-200,200{]}}}) -- 
\emph{{[}OPTIONAL{]}}  (Excerpt of Firmware-comments, in c\_output.c):
``Turn ratio is how tight you turn and to what direction you turn.
\begin{itemize}
\item {} 
0 value is moving straight forward

\item {} 
Negative values turn to the left

\item {} 
Positive values turn to the right

\item {} 
Value -100 stops the left motor

\item {} 
Value +100 stops the right motor

\item {} 
Values less than -100 makes the left motor run the opposite direction of the right motor (Spin)

\item {} 
Values greater than +100 makes the right motor run the opposite direction of the left motor (Spin)''

\end{itemize}


\end{itemize}

\end{description}\end{quote}
\paragraph{Notes}
\begin{itemize}
\item {} 
This is right now a pretty `heavy' function, as it tests if both motors are
connected AND aren't running, wasting four packets, keep that in mind

\item {} 
It is necessary to call syncedStop() and not stop() for stopping the motors
(otherwise the sync-state cannot be exited correctly)

\end{itemize}
\paragraph{Example}

b = EV3(); 
b.connect(`usb'); 
m = b.motorA; 
slave = b.motorB; 
m.power = 50; 
m.syncedStart(slave); 
\% Do stuff
m.syncedStop(); 

\end{fulllineitems}

\index{syncedStop() (source.Motor method)}

\begin{fulllineitems}
\phantomsection\label{source:source.Motor.syncedStop}\pysiglinewithargsret{\sphinxbfcode{syncedStop}}{\emph{motor}}{}
Stops both motors previously started with syncedStart.

See also MOTOR.SYNCEDSTART

\end{fulllineitems}

\index{waitFor() (source.Motor method)}

\begin{fulllineitems}
\phantomsection\label{source:source.Motor.waitFor}\pysiglinewithargsret{\sphinxbfcode{waitFor}}{\emph{motor}}{}
Stops execution of program as long as motor is running
\paragraph{Notes}
\begin{itemize}
\item {} 
(OLD)This one's a bit tricky. The opCode OutputReady makes the brick stop sending
responses until the motor has stopped. For security reasons, in this toolbox
there is an internal timeout for receiving messages from the brick. It raises
an error if a reply takes too long, which would happen in this case. As a
workaround, there is an infinite loop that catches errors from outputReady and
continues then, until outputReady will actually finish without an error.

\item {} 
(OLD)OutputReady (like OutputTest in isRunning) sometimes doesn't work. If
outputReady returns in less than a second, another while-loop iterates until
the motor has stopped, this time using motor.isRunning() (this only works as
long as not both OutputTest and OutputReady are buggy).

\item {} 
(OLD)Workaround: Poll isRunning (which itself return (speed\textgreater{}0)) until it
is false (No need to check if motor is connected as speed correctly
returns 0 if it's not)

\end{itemize}

\end{fulllineitems}


\end{fulllineitems}



\chapter{Sensor}
\label{source:sensor}\index{Sensor (class in source)}

\begin{fulllineitems}
\phantomsection\label{source:source.Sensor}\pysiglinewithargsret{\sphinxstrong{class }\sphinxcode{source.}\sphinxbfcode{Sensor}}{\emph{varargin}}{}
High-level class to work with sensors.

The Sensor-class facilitates the communication with sensors. This mainly consists of
reading the sensor's type and current value in a specified mode.
\paragraph{Notes}
\begin{itemize}
\item {} 
You don't need to create instances of this class. The EV3-class automatically creates
instances for each sensor port, and you can work with them via the EV3-object.

\item {} 
The Sensor-class represents sensor ports, not individual sensors!

\end{itemize}
\index{mode (source.Sensor attribute)}

\begin{fulllineitems}
\phantomsection\label{source:source.Sensor.mode}\pysigline{\sphinxbfcode{mode}}
\emph{DeviceMode.\{Type\}} -- Sensor mode in which the value will be read. By default,
mode is set to DeviceMode.Default.Undefined. Once a physical sensor is connected
to the port \emph{and} the physical Brick is connected to the EV3-object, the allowed
mode and the default mode for a Sensor-object are the following (depending on the
sensor type): \emph{{[}WRITABLE{]}}
\begin{itemize}
\item {} \begin{description}
\item[{Touch-Sensor:}] \leavevmode\begin{itemize}
\item {} 
DeviceMode.Touch.Pushed {[}Default{]}

\item {} 
DeviceMode.Touch.Bumps

\end{itemize}

\end{description}

\item {} \begin{description}
\item[{Ultrasonic-Sensor:}] \leavevmode\begin{itemize}
\item {} 
DeviceMode.UltraSonic.DistCM {[}Default{]}

\item {} 
DeviceMode.UltraSonic.DistIn

\item {} 
DeviceMode.UltraSonic.Listen

\end{itemize}

\end{description}

\item {} \begin{description}
\item[{Color-Sensor:}] \leavevmode\begin{itemize}
\item {} 
DeviceMode.Color.Reflect {[}Default{]}

\item {} 
DeviceMode.Color.Ambient

\item {} 
DeviceMode.Color.Col

\end{itemize}

\end{description}

\item {} \begin{description}
\item[{Gyro-Sensor:}] \leavevmode\begin{itemize}
\item {} 
DeviceMode.Gyro.Angular {[}Default{]}

\item {} 
DeviceMode.Gyro.Rate

\end{itemize}

\end{description}

\end{itemize}

\end{fulllineitems}

\index{debug (source.Sensor attribute)}

\begin{fulllineitems}
\phantomsection\label{source:source.Sensor.debug}\pysigline{\sphinxbfcode{debug}}
\emph{bool} -- Debug turned on or off. In debug mode, everytime a command is passed to
the sublayer (`communication layer'), there is feedback in the console about what
command has been called. \emph{{[}WRITABLE{]}}

\end{fulllineitems}

\index{value (source.Sensor attribute)}

\begin{fulllineitems}
\phantomsection\label{source:source.Sensor.value}\pysigline{\sphinxbfcode{value}}
\emph{numeric} -- Value read from hysical sensor. What the value represents depends on
sensor.mode. \emph{{[}READ-ONLY{]}}

\end{fulllineitems}

\index{type (source.Sensor attribute)}

\begin{fulllineitems}
\phantomsection\label{source:source.Sensor.type}\pysigline{\sphinxbfcode{type}}
\emph{DeviceType} -- Type of physical sensor connected to the port. Possible types are: {[}READ-ONLY{]}
\begin{itemize}
\item {} 
DeviceType.NXTTouch

\item {} 
DeviceType.NXTLight

\item {} 
DeviceType.NXTSound

\item {} 
DeviceType.NXTColor

\item {} 
DeviceType.NXTUltraSonic

\item {} 
DeviceType.NXTTemperature

\item {} 
DeviceType.LargeMotor

\item {} 
DeviceType.MediumMotor

\item {} 
DeviceType.Touch

\item {} 
DeviceType.Color

\item {} 
DeviceType.UltraSonic

\item {} 
DeviceType.Gyro

\item {} 
DeviceType.InfraRed

\item {} 
DeviceType.Unknown

\item {} 
DeviceType.None

\item {} 
DeviceType.Error

\end{itemize}

\end{fulllineitems}

\index{reset() (source.Sensor method)}

\begin{fulllineitems}
\phantomsection\label{source:source.Sensor.reset}\pysiglinewithargsret{\sphinxbfcode{reset}}{\emph{sensor}}{}
Resets value on sensor
\paragraph{Notes}
\begin{itemize}
\item {} 
This clears ALL the sensors right now, no other Op-Code available... :(

\end{itemize}

\end{fulllineitems}

\index{setProperties() (source.Sensor method)}

\begin{fulllineitems}
\phantomsection\label{source:source.Sensor.setProperties}\pysiglinewithargsret{\sphinxbfcode{setProperties}}{\emph{sensor}, \emph{varargin}}{}
Sets multiple Sensor properties at once using MATLAB's inputParser.
\begin{quote}\begin{description}
\item[{Parameters}] \leavevmode\begin{itemize}
\item {} 
\textbf{\texttt{debug}} (\emph{\texttt{bool}}) -- 

\item {} 
\textbf{\texttt{mode}} (\emph{\texttt{DeviceMode.\{Type\}}}) -- 

\end{itemize}

\end{description}\end{quote}
\paragraph{Example}

b = EV3(); 
b.connect(`bt', `serPort', `/dev/rfcomm0'); 
b.sensor1.setProperties(`debug', `on', `mode', DeviceMode.Color.Ambient); 
\% Instead of: b.sensor1.debug = `on'; 
\%             b.sensor1.mode = DeviceMode.Color.Ambient; 

\end{fulllineitems}


\end{fulllineitems}



\chapter{Indices and tables}
\label{index:indices-and-tables}\begin{itemize}
\item {} 
\emph{genindex}

\item {} 
\emph{modindex}

\item {} 
\emph{search}

\end{itemize}


\renewcommand{\indexname}{MATLAB Module Index}
\begin{theindex}
\def\bigletter#1{{\Large\sffamily#1}\nopagebreak\vspace{1mm}}
\bigletter{s}
\item {\texttt{source}}, \pageref{source:module-source}
\end{theindex}

\renewcommand{\indexname}{Index}
\printindex
\end{document}
